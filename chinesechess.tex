\documentclass[full]{l3doc}
\usepackage[scheme=chinese]{ctex}
\usepackage{enumitem}
\usepackage{indentfirst}
\usepackage{titling}
\usepackage{geometry}
\usepackage{graphicx}
\usepackage{fontawesome5}
\usepackage{fancyvrb-ex}
\usepackage[piecechar={C}{炮}]{chinesechess}
% \usepackage[draft,piecechar={C}{炮}]{chinesechess}

\IndexPrologue
  {
    \section*{Index}
    \markboth{Index}{Index}
    \addcontentsline{toc}{section}{Index}
    The~italic~numbers~denote~the~pages~where~the~
    corresponding~entry~is~described,~
    numbers~underlined~point~to~the~definition,~
    all~others~indicate~the~places~where~it~is~used.
  }

\newcommand\tikzmark[1]{\tikz \coordinate[overlay, remember picture] (#1);}

\geometry{
  left=4.5cm,
  right=2cm,
  top=2cm,
  bottom=2cm,
}
\hypersetup {
  CJKbookmarks,
  bookmarksopen,
  bookmarksopenlevel=3,
  pdfstartview=FitH,
  pdfinfo = {
   Title = 中国象棋排版宏包 ,
   Subject = A LaTeX3 package ,
   Author = 耿楠
 }
}

\DoNotIndex{\begin, \end}
\setlength{\parskip}{\medskipamount}
\DeclareDocumentEnvironment { noteen } { +b } {
  \par\textbf{\textsf{NOTE:~}}#1\par
} {}
\DeclareDocumentEnvironment { notezh } { +b } {
  \par\textbf{\textsf{注意:~}}#1\par
} {}

\AtEndDocument{
  \newgeometry{
    left=2cm,
    right=2cm,
    top=2cm,
    bottom=2cm
  }
  \PrintIndex
}

\ExplSyntaxOn
\dim_new:N \l__my_syntax_dim
\box_new:N \g__my_syntax_box
\NewDocumentEnvironment { Syntax } { s }
  {
    \dim_set:Nn \l__my_syntax_dim
      { \textwidth }
    \hbox_gset:Nw \g__my_syntax_box
      \small \ttfamily
      \begin{minipage}[t]{\l__my_syntax_dim}
        \raggedright\obeyspaces\obeylines
  }
  {
      \end{minipage}
    \hbox_gset_end:
    \IfValueF { #1 } { \smallskip }
    \box_use_drop:N \g__my_syntax_box
    \smallskip
  }

\DeclareDocumentEnvironment { Description } { o +b } {
  \hbox_set:Nn \l_tmpa_box { #1 }
  \dim_set:Nn \l_tmpa_dim { \box_wd:N \l_tmpa_box }
  \begin{itemize}[labelwidth=\l_tmpa_dim, align=left]
    #2
  \end{itemize}
} {  }

\keys_define:nn { cchess/doc } {
  opt .tl_set:N = \l_opt_tl,
  desc .tl_set:N = \l_desc_tl,
  init .tl_set:N = \l_init_tl,
  init .initial:n = init-none,
}

\box_new:N \l__option_box
\NewDocumentEnvironment { option } { m +b } {
  \keys_set:nn { cchess/doc } { #1 }
  \hbox_set:Nw \l__option_box
    \small \ttfamily
    \begin{minipage}[t]{\textwidth}
      \obeyspaces\obeylines
      \textcolor{red}{
        \l_opt_tl
        \exp_args:Nx\SpecialOptionIndex{\l_opt_tl}
      }
      {~}\l_desc_tl
      \hfill(
      \tl_if_eq:NnTF \l_init_tl { init-none } { no~value }
        { 初始值:~\texttt{\l_init_tl} }
      )
    \end{minipage}
  \hbox_gset_end:
  \box_use_drop:N \l__option_box
  #2
  % \vspace*{-4ex}
  \medskip
} {  }

\DeclareDocumentCommand \opt { O{} m }
  { \__codedoc_cmd:no {#1} { #2 } }
\ExplSyntaxOff

\def\vers{\texttt{v1.2.0} }

\begin{document}
\title{
  中国象棋棋谱排版\\\pkg{chinesechess} 宏包
  % \rlap{\makebox[4cm][r]{
    % \normalsize $\Longrightarrow$ \color{red}
    % \protect\hyperlink{en}{English Version}
    % \protect\hypertarget{zh}{}
  % }}
}
\author{\textit{耿楠} \texttt{<nangeng@nwafu.edu.cn>}}
\date{\the\year 年\the\month 月\the\day 日\qquad \vers
% \thanks{\url{https://github.com/registor/chinesechess}}
\thanks{\url{https://gitee.com/nwafu_nan/chinesechess}}
}

\maketitle

\begin{abstract}
  \pkg{chinesechess}是一个基于\pkg{l3draw}绘图宏包,用\pkg{Expl3}%
  开发的中国象棋排版宏包。它提供了\tn{cchessboard}(全棋盘)、%
  \tn{cchessbord*}(空棋盘)和\tn{cchessman}(棋谱)排版命令。同时,
  该宏包还提供了用于打谱的排版环境\env{setcchessman},在该环境
  中,可以通过环境专用命令\tn{init}(初始化棋谱状态),\tn{set}(在
  指定位置布置棋子),\tn{del}(删除指定位置的棋子),\tn{mov}(实现
  棋子的移动)实现打谱操作,并在环境结束后排版最终状态棋谱。
  通过其星号环境\env{setcchessman*}可以记录打谱过程,并通过%
  \tn{printman}利用交叉引用标签\oarg{label}选项对指定%
  \env{setcchessman*}打谱环境记录的打谱文本在指定位置排版输出。
  棋子、棋盘、背景等外观可以通过命令、环境选项或\tn{cchessset}%
  命令进行设置。
\end{abstract}

{\small
\tableofcontents
}
\newpage

\begin{documentation}

\section{引言}

\pkg{chinesechess}是一个基于\pkg{l3draw}绘图宏包,用\pkg{Expl3}%
开发的中国象棋排版宏包。它利用\pkg{l3draw}的绘图功能,通过为红黑
各7个棋子文字添加背景实现了棋子绘制,利用简单线条实现了棋盘绘制,
并可以通过\pkg{graphicx}宏包可以为棋盘添加图片背景。

该宏包的目的是实现\textsf{不依赖于字体文件}或\textsf{字体宏包}的
中国象棋棋谱排版,由于需要大量绘制操作,因此,其编译速度较低。

\section{用户接口}

\subsection{\tn{cchessboard}棋盘排版命令}

\begin{function}{\cchessboard}
  \begin{syntax}
    \cs{cchessboard} \oarg{外观选项}
    \cs{cchessboard*} \oarg{外观选项}
  \end{syntax}
\end{function}

  按设置的\oarg{外观选项}输出带有32个红黑棋子,并布置于初始位置
  的棋盘。

  在\oarg{外观选项}中可以通过key-value的方式设置棋子、棋盘的颜色、
  字体、字号、线宽等外观。

  通过\oarg{外观选项}设置的外观参数仅对当前命令局部有效,

  其星号版本命令用于无棋子棋盘排版。

\begin{SideBySideExample}[frame=single,numbers=left,
                xrightmargin=.60\linewidth,gobble=2]
  \centering
  % 为便排版,进行缩放
  \cchessset{resize/type=real,
          resize/height=4.0cm}
  \cchessboard\quad
  \cchessboard*
\end{SideBySideExample}

\subsection{\tn{cchessman}棋谱排版命令}\label{subsec-cchessman}

\begin{function}{\cchessman}
  \begin{syntax}
    \cs{cchessman} \oarg{外观选项} \marg{棋子列表}
  \end{syntax}
\end{function}

  按设置的\oarg{外观选项},将在\marg{棋子列表}中指定棋子布置于棋盘的
  指定位置并排版输出。

  \marg{棋子列表}中各个棋子用大括号分组后再用英文逗号进行分隔,每个棋子
  是由大括号括起来的\textsf{棋子位置}与\textsf{棋子编码}组成的。

  棋子位置由英文逗号分隔的纵横坐标确定,其中,纵向坐标用英文小写字母
  a--j或阿拉伯数字0--9分别表示棋盘纵向10个交点;横向坐标用英文小写字母
  a--i或阿拉伯数字0--8分别表示棋盘横向的9个交点。

  \textbf{\textsf{注意: }}可以混合使用字母和数字棋子坐标表示方法,但强烈
  建议使用单一字母法或数字法表示棋子坐标。

  棋子编码采用大写字母表示红棋,小写字母表示黑棋,具体如表%
  \ref{tab-piececode}所示。

  \begin{table}[!htbp]
    \centering
    \caption{棋子编码规则}\label{tab-piececode}
    \begin{tabular}{cc||cc}
      \hline
      编码 & 红棋 & 编码 & 黑棋\\\hline
      K    & 帥   & k    & 將  \\
      A    & 仕   & a    & 士  \\
      B(E) & 相   & b(e) & 象  \\
      N(H) & 马   & n(h) & 馬  \\
      R    & 车   & r    & 車  \\
      C    & 炮   & c    & 炮  \\
      P    & 兵   & p    & 卒  \\\hline
    \end{tabular}
  \end{table}

  例如,一个完整的\marg{棋子列表}为:

  |{{4,0}{K}}, {{2,2}{p}}, {{1,7}{C}}, {{4,9}{k}}|

  在\oarg{外观选项}中可以通过key-value的方式设置棋子、棋盘的颜色、
  字体、字号、线宽等外观。

  通过\oarg{外观选项}设置的外观参数仅对当前命令局部有效,

\begin{SideBySideExample}[frame=single,numbers=left,
                xrightmargin=.50\linewidth,gobble=2]
  \centering
  % 为便排版,进行缩放
  \cchessset{resize/type=real,
          resize/width=.90\linewidth}
  \cchessman{ {{4,0}{K}}, {{2,2}{p}},
              {{1,7}{C}}, {{4,9}{k}}
            }
\end{SideBySideExample}

\subsection{\env{setcchessman}打谱排版环境}

\begin{function}{setcchessman}
  \begin{syntax}
    \tn{begin}|{setcchessman}|\oarg{外观选项}
    .....
    \tn{end}|{setcchessman}|
    \tn{begin}|{setcchessman*}|\oarg{外观选项}
    .....
    \tn{end}|{setcchessman*}|
  \end{syntax}
\end{function}

  按设置的\oarg{外观选项}排版打谱结果。

  在\env{setcchessman}环境中,可以通过专用命令\tn{init}%
  (初始化棋谱状态)(残局),\tn{set}(在指定位置布置棋子),\tn{del}(删除
  指定位置的棋子),\tn{mov}(棋子移动)实现打谱操作。
  在环境结束后以最终打谱状态排版棋谱。

  星号环境\env{setcchessman*}用于同时记录打谱过程的文本描述(如:车一进二等)。
  在\env{setcchessman*}环境中可以用\oarg{label}选项设置引用标签,
  再用\tn{printman}命令通过引用标签输出对应打谱过程文本描述。

  在\oarg{外观选项}中可以通过key-value的方式设置棋子、棋盘的颜色、
  字体、字号、线宽等外观。对于\env{setcchessman*}星号环境,%
  \oarg{外观选项}中的棋子与棋盘相关外观选项设置无效。

  通过\oarg{外观选项}设置的外观参数仅对当前命令局部有效,

\begin{SideBySideExample}[frame=single,numbers=left,
                xrightmargin=.48\linewidth,gobble=2]
  \centering
  % 为便排版,进行缩放
  \cchessset{resize/type=real,
        resize/width=.90\linewidth}
  \begin{setcchessman}
    % 残局初始化
    \init{ {{4,0}{K}}, {{2,2}{p}},
           {{1,7}{C}}, {{4,9}{k}} }
    % 添加棋子
    \set{e,d}{P} \set{8,g}{h}
    % 删除棋子
    \del{4,3}
    % 移动棋子
    \mov{C}{b,h}{b,b} \mov{p}{2,2}{3,2}
    \mov{K}{4,0}{4,1} \mov{h}{8,6}{6,7}
    \mov{K}{4,1}{3,1} \mov{h}{6,7}{5,5}
  \end{setcchessman}
\end{SideBySideExample}

\subsection{打谱专用命令}

\subsubsection{\tn{init}棋谱初始化命令}

\begin{function}{\init}
  \begin{syntax}
    \cs{init} \marg{棋子列表}
  \end{syntax}
\end{function}

  用于将\marg{棋子列表}中指定的所有棋子布置于棋盘的指定位置。

  \marg{棋子列表}中的棋子表示方法与第\ref{subsec-cchessman}小节
  中的说明相同。

  \textbf{\textsf{注意: }}该命令只能用于\env{setcchessman}环境中。

\subsubsection{\tn{set}棋子布置命令}

\begin{function}{\set}
  \begin{syntax}
    \cs{set} \marg{棋子}
  \end{syntax}
\end{function}

  用于将指定\marg{棋子}布置于棋盘的指定位置。

  \marg{棋子}中的棋子表示方法与第\ref{subsec-cchessman}小节
  中的说明相同。

  \textbf{\textsf{注意: }}该命令只能用于\env{setcchessman}环境中。

\subsubsection{\tn{del}棋子删除命令}

\begin{function}{\del}
  \begin{syntax}
    \cs{del} \marg{棋盘坐标}
  \end{syntax}
\end{function}

  用于删除\marg{棋盘坐标}中指定坐标处的棋子。

  \marg{棋盘坐标}中的表示方法与第\ref{subsec-cchessman}小节
  中的说明相同。

  \textbf{\textsf{注意: }}该命令只能用于\env{setcchessman}环境中。

\subsubsection{\tn{mov}棋子移动命令}

\begin{function}{\mov}
  \begin{syntax}
    \cs{mov} \marg{棋子编码}\marg{原棋盘坐标}\marg{新棋盘坐标}
  \end{syntax}
\end{function}

  用于将\marg{棋子编码}中指定的棋子从\marg{原棋盘坐标}移动到%
  \marg{新棋盘坐标}。若\marg{新棋盘坐标}处有棋子,则会被无条件
  ``吃''掉(替换)。

  \marg{棋子编码}、\marg{原棋盘坐标}和\marg{新棋盘坐标}中采用
  棋子与棋盘坐标表示方法与第\ref{subsec-cchessman}小节中的说明相同。

  \textbf{\textsf{注意: }}该命令只能用于\env{setcchessman}环境中。

\subsection{\tn{printman}输出打谱结果命令}

\begin{function}{\printman}
  \begin{syntax}
    \cs{printman} \oarg{每行步数} \marg{label标签}
  \end{syntax}
\end{function}

  输出由label标签指定的\env{setcchessman*}打谱环境生成的棋谱文本描述。
  \footnote{棋谱文本描述文件以\tn{jobname}+label+索引编号命名,
            并以``.man''为后缀名,是一个纯文本文件。}

  \oarg{每行步数}选项用于指定每行中输出的棋谱步数。

\subsection{\tn{getpiece}输出棋子命令}

\begin{function}{\getpiece}
  \begin{syntax}
    \cs{getpiece} \oarg{棋子字号} \marg{棋子编码}
  \end{syntax}
\end{function}

  按\oarg{棋子字号}指定的字号将用\marg{棋子编码}指定的棋子在当前位置输出。

  \marg{棋子编码}中的编码详见表\ref{tab-piececode}。

\begin{SideBySideExample}[frame=single,numbers=left,
                xrightmargin=.28\linewidth,gobble=2]
  \centering
  \getpiece{K}\getpiece{A}\getpiece{B}\getpiece{R}%
  \getpiece{N}\getpiece{C}\getpiece{P}

  \getpiece{k}\getpiece{a}\getpiece{b}\getpiece{r}%
  \getpiece{n}\getpiece{c}\getpiece{p}
\end{SideBySideExample}

\subsection{\tn{piecechar}棋子字符设置命令}

\begin{function}{\piecechar}
  \begin{syntax}
    \cs{piecechar} \marg{棋子编码} \marg{棋子字符}
  \end{syntax}
\end{function}

  用于将\marg{棋子编码}指定的棋子字符设置为指定的\marg{棋子字符}。

  \marg{棋子编码}中的编码详见表\ref{tab-piececode}。

  \textbf{\textsf{说明: }}该功能也可以通过|piecechar|外观选项实现。

\subsection{\tn{resetpiece}棋子字符复位命令}

\begin{function}{\resetpiece}
  \begin{syntax}
    \cs{resetpiece}
  \end{syntax}
\end{function}

  将棋子字符复位到默认字符。

\subsection{\tn{cchessset}选项设置命令}

\begin{function}{\cchessset}
  \begin{syntax}
    \cs{cchessset} \marg{外观选项}
  \end{syntax}
\end{function}

  通过\marg{外观选项}中的key-value方式设置棋子、棋盘的文字、
  字体、颜色、线宽等外观参数。

  % \textbf{\textsf{说明: }}执行该命令时,仅有\textbf{\textsf{除}}%
  % 棋子与棋盘外观的\textbf{\textsf{其它参数}}有效,棋子与棋盘外观
  % 参数的设置会在执行下一次命令或环境参数时起效。

  通过\cs{cchessset}\marg{外观选项}设置的外观参数对后续所有
  棋谱排版命令有效。

  \textbf{\textsf{注意: }}由于\pkg{chinesechess}宏包涉及大量
  绘图操作,因此建议尽量减少在命令或环境中使用\oarg{外观选项}设置%
  以节约编译时间。如需要更改棋子或棋盘外观,可以在引用宏包时,
  通过为宏包添加选项实现,也可以执行\tn{cchessset}命令进行必要的
  全局设置或在一定范围内进行统一设置。

  % \oarg{外观选项}中的棋子与棋盘相关外观选项设置无效。

% \section{宏包选项}

\section{宏包选项}

在\pkg{chinesechess}宏包中,棋子与棋盘的文本、字体、颜色、背景等
外观可以可以在引入宏包时通过\oarg{宏包选项}进行设置,也可以在
命令或环境的\oarg{外观选项}中进行局部设置,还可以通过\cs{cchessset}命令
进行全局设置。

强烈建议在引用宏包时通过\oarg{宏包选项}为一个文档统一全局设置棋子与
棋盘外观,以节约编译时间。应避免频繁在命令和环境中使用\oarg{宏包选项}或
使用\cs{cchessset}命令设置棋子与棋盘外面。

\pkg{chinesechess}宏包的选项是一组由(英文)逗号分隔的选项列表,列表中的
选项通常是\marg{key}|=|\marg{value}形式。部分选项的\marg{value}可以省略。
对于同一选项,后续设置会覆盖以前的设置。

\pkg{chinesechess}采用\LaTeX3风格的键值设置,支持不同类型以及多种层次的
选项设定。键值列表中,``|=|''左右的空格不影响设置。但需注意,参数列表中%
\textsf{不可以出现空行}。

布尔型的参数\marg{选项}|=true|中的``|=true|''可以省略。

对于``|resize|''缩放和``|man|''子选项,只能用于命令与环境选项中,或通过
\tn{cchessset}命令进行设置。

\subsection{草稿选项}

\begin{option}{ opt = draft, desc = {= \meta{草稿选项}}, init=false }
  设置草稿选项。
\end{option}

  \oarg{draft}选项将删除所有效果设置,所有棋子仅显示文本,
  棋盘仅带有九宫格对角线。从而加快编译速度。

  \textbf{\textsf{注意: }}在草稿模式下,命令与环境中与棋子与棋盘外观
  相关的选项都将失效,因此,可能与最终排版结果会有出入。

\begin{SideBySideExample}[frame=single,numbers=left,
                xrightmargin=.53\linewidth,gobble=2]
  % \usepackage[draft]{chinesechess}
  \centering
  % 为便排版,进行缩放
  \cchessset{resize/type=real,
     resize/width=.90\linewidth}
  \cchessboard[draft]
\end{SideBySideExample}
\subsection{棋盘}

\subsubsection{单元格尺寸}

\begin{option}{ opt = gridsize, desc = {= \meta{棋盘格尺寸}}, init=10mm }
  设置棋盘单元格尺寸。
\end{option}

\begin{SideBySideExample}[frame=single,numbers=left,
                xrightmargin=.60\linewidth,gobble=2]
  \centering
  % 为便排版,进行缩放
  \cchessset{resize/type=real,
    resize/width=.90\linewidth}
  \cchessboard[gridsize=15mm]
\end{SideBySideExample}

\bigskip

\subsubsection{棋盘类型}

\begin{option}{ opt = boardtype, desc = {= \meta{棋盘类型}}, init=x+tn }
  设置棋盘类型,目前支持:
\end{option}

  \begin{Description}[a]
    \item |x|---有九宫对角线,无兵炮位置标志。
    \item |x+|---有九宫对角线和兵炮位置标志。
    \item |x+t|---有九宫对角线、兵炮位置标志和楚河汉界。
    \item |x+tn|---有九宫对角线、兵炮位置标志、楚河汉界和纵线标位数字。
    \item |x+Xtn|---有九宫对角线、兵炮位置标志、楚河汉界、纵线标位数字和象位虚线。
  \end{Description}

\begin{SideBySideExample}[frame=single,numbers=left,
                xrightmargin=.54\linewidth,gobble=2]
  \centering
  % 为便排版,进行缩放
  \cchessset{resize/type=real,
        resize/width=.35\linewidth}
  \cchessboard*[boardtype=x]\quad
  \cchessboard*[boardtype=x+]\\
  \cchessboard*[boardtype=x+t]\quad
  \cchessboard*[boardtype=x+tn]\\
  \cchessboard*[boardtype=x+Xtn]
\end{SideBySideExample}

\bigskip

\subsubsection{棋盘线线宽}

\begin{option}{ opt = boardlinewd, desc = {= \meta{棋盘线宽}}, init=0.4pt }
  设置棋盘线线宽。
\end{option}

\begin{SideBySideExample}[frame=single,numbers=left,
                xrightmargin=.56\linewidth,gobble=2]
  \centering
  % 为便排版,进行缩放
  \cchessset{resize/type=real,
     resize/width=.90\linewidth}
  \cchessboard*[boardlinewd=1.0pt]
\end{SideBySideExample}

\bigskip

\subsubsection{棋盘线颜色}

\begin{option}{ opt = boardlinecolor, desc = {= \meta{棋盘线颜色}}, init=black }
  设置棋盘线颜色。
\end{option}

\begin{SideBySideExample}[frame=single,numbers=left,
                xrightmargin=.55\linewidth,gobble=2]
  \centering
  % 为便排版,进行缩放
  \cchessset{resize/type=real,
    resize/width=.90\linewidth}
  \cchessboard*[boardlinecolor=red]
\end{SideBySideExample}

\bigskip

\subsubsection{棋盘背景}

\begin{option}{ opt = boardbg, desc = {= \meta{棋盘背景}}, init=init-none }
  设置棋盘背景图的插图文件路径及文件名。
\end{option}

  \textbf{\textsf{注意: }}使用该选项时,需要在导言区添加|\usepackage{grpahicx}|命令。

\begin{SideBySideExample}[frame=single,numbers=left,
                xrightmargin=.55\linewidth,gobble=2]
  \centering
  % 为便排版,进行缩放
  \cchessset{resize/type=real,
     resize/width=.35\linewidth}
  \cchessboard*[boardbg=bg01]\quad
  \cchessboard*[boardbg=bg02]
\end{SideBySideExample}

\bigskip

\subsection{棋子}

\subsubsection{字符}

\begin{option}{ opt = piecechar, desc = {= \meta{棋子字符}},
                init=\{K\}\{帥\} }
  设置棋子字符,用两组大括号分别指定棋子编码和棋子字符。
  其中,棋子编码见表\ref{tab-piececode}。
\end{option}

  \textbf{\textsf{注意: }}使用该选项时,在能需要在导言区添加字体,
  如|\usepackage{fontawesome5}|命令。

\begin{SideBySideExample}[frame=single,numbers=left,
                xrightmargin=.55\linewidth,gobble=2]
  %\usepackage{fontawesome5}
  \centering
  % 为便排版,进行缩放
  \cchessset{resize/type=real,
    resize/width=.90\linewidth}
  \cchessboard[%
    piecechar={K}{\faChessKing},
    piecechar={A}{\faShield*},
    piecechar={E}{\faChessBishop},
    piecechar={H}{\faChessKnight},
    piecechar={R}{\faChessRook},
    piecechar={C}{\faMeteor},
    piecechar={P}{\faChessPawn},
    piecechar={k}{\faChessQueen},
    piecechar={a}{\faShield*},
    piecechar={e}{\faChessBishop},
    piecechar={h}{\faChessKnight},
    piecechar={r}{\faChessRook},
    piecechar={c}{\faMeteor},
    piecechar={p}{\faChessPawn},
    boardlinecolor=red]
\end{SideBySideExample}

\bigskip

\resetpiece
\cchessset{piecechar={C}{炮}}

\subsubsection{字体}

\begin{option}{ opt = piecefont, desc = {= \meta{棋子字符字体}}, init=\cs{kaishu} }
  设置棋子字符字体。
\end{option}

\begin{SideBySideExample}[frame=single,numbers=left,
                xrightmargin=.54\linewidth,gobble=2]
  \centering
  % 为便排版,进行缩放
  \cchessset{resize/type=real,
        resize/width=.90\linewidth}
  \cchessboard[piecefont=\sffamily]
\end{SideBySideExample}

\bigskip

\subsubsection{边框类型}

\begin{option}{ opt = piecetype, desc = {= \meta{棋子边框类型}}, init=|ooo| }
  设置棋子外框类型,目前支持:
\end{option}

  \begin{Description}[a]
    \item |o|---实心填充圆(小写英文字母``|o|'')。
    \item |oo|---双同心实心圆。
    \item |ooo|---双同心实心圆叠加阴影错位圆。
  \end{Description}

\begin{SideBySideExample}[frame=single,numbers=left,
                xrightmargin=.55\linewidth,gobble=2]
  \centering
  % 为便排版,进行缩放
  \cchessset{resize/type=real,
     resize/width=.40\linewidth}
  \cchessboard[piecetype=o]\quad
  \cchessboard[piecetype=oo]\\
  \cchessboard[piecetype=ooo]
\end{SideBySideExample}

\bigskip

\subsubsection{边框线宽}

\begin{option}{ opt = boxlinewd, desc = {= \meta{棋子外框线宽度}}, init=0.4pt }
  设置棋子外框线线宽。
\end{option}

\begin{SideBySideExample}[frame=single,numbers=left,
                xrightmargin=.60\linewidth,gobble=2]
  \centering
  % 为便排版,进行缩放
  \cchessset{resize/type=real,
    resize/width=.95\linewidth}
  \cchessboard[boxlinewd=0.8pt]
\end{SideBySideExample}

\bigskip

\subsubsection{边框颜色}

\begin{option}{ opt = boxcolor, desc = {= \meta{棋子外框线颜色}}, init=black }
  设置棋子外框线颜色。
\end{option}

\begin{SideBySideExample}[frame=single,numbers=left,
                xrightmargin=.60\linewidth,gobble=2]
  \centering
  % 为便排版,进行缩放
  \cchessset{resize/type=real,
    resize/width=.95\linewidth}
  \cchessboard[boxcolor=white]
\end{SideBySideExample}

\bigskip

\subsubsection{红棋字符颜色}

\begin{option}{ opt = redpiece, desc = {= \meta{红棋棋子字符颜色}}, init=red }
  设置红棋棋子字符颜色。
\end{option}

\begin{SideBySideExample}[frame=single,numbers=left,
                xrightmargin=.60\linewidth,gobble=2]
  \centering
  % 为便排版,进行缩放
  \cchessset{resize/type=real,
    resize/width=.95\linewidth}
  \cchessboard[redpiece=white]
\end{SideBySideExample}

\bigskip

\subsubsection{黑棋字符颜色}

\begin{option}{ opt = blkpiece, desc = {= \meta{黑棋棋子字符颜色}}, init=black }
  设置黑棋棋子字符颜色。
\end{option}

\begin{SideBySideExample}[frame=single,numbers=left,
                xrightmargin=.60\linewidth,gobble=2]
  \centering
  % 为便排版,进行缩放
  \cchessset{resize/type=real,
    resize/width=.95\linewidth}
  \cchessboard[blkpiece=white]
\end{SideBySideExample}

\bigskip

\subsubsection{背景圆填充色}

\begin{option}{ opt = lower, desc = {= \meta{棋子背景圆填充色}}, init=init-none }
  设置棋子背景圆填充色。
\end{option}

\begin{SideBySideExample}[frame=single,numbers=left,
                xrightmargin=.60\linewidth,gobble=2]
  \centering
  % 为便排版,进行缩放
  \cchessset{resize/type=real,
    resize/width=.95\linewidth}
  \cchessboard[lower=black]
\end{SideBySideExample}

\bigskip

\subsubsection{表面圆环填充色}

\begin{option}{ opt = donut, desc = {= \meta{棋子表面圆环填充色}}, init=init-none }
  设置棋子表面圆环填充色。
\end{option}

\begin{SideBySideExample}[frame=single,numbers=left,
                xrightmargin=.60\linewidth,gobble=2]
  \centering
  % 为便排版,进行缩放
  \cchessset{resize/type=real,
    resize/width=.95\linewidth}
  \cchessboard[donut=red]
\end{SideBySideExample}

\bigskip

\subsubsection{红棋表面圆填充色}

\begin{option}{ opt = redupper, desc = {= \meta{红棋棋子表面圆填充色}}, init=init-none }
  设置红棋棋子表面圆填充色。
\end{option}

\begin{SideBySideExample}[frame=single,numbers=left,
                xrightmargin=.60\linewidth,gobble=2]
  \centering
  % 为便排版,进行缩放
  \cchessset{resize/type=real,
    resize/width=.95\linewidth}
  \cchessboard[redupper=red,
             redpiece=white]
\end{SideBySideExample}

\bigskip

\subsubsection{黑棋表面圆填充色}

\begin{option}{ opt = blkupper, desc = {= \meta{黑棋棋子表面圆填充色}}, init=init-none }
  设置黑棋棋子表面圆填充色。
\end{option}

\begin{SideBySideExample}[frame=single,numbers=left,
                xrightmargin=.60\linewidth,gobble=2]
  \centering
  % 为便排版,进行缩放
  \cchessset{resize/type=real,
    resize/width=.95\linewidth}
  \cchessboard[blkupper=black,
               blkpiece=white]
\end{SideBySideExample}

\bigskip

\subsubsection{阴影错位圆填充色}

\begin{option}{ opt = shadow, desc = {= \meta{棋子阴影错位圆填充色}}, init=init-none }
  设置棋子阴影错位圆填充色。
\end{option}

\begin{SideBySideExample}[frame=single,numbers=left,
                xrightmargin=.60\linewidth,gobble=2]
  \centering
  % 为便排版,进行缩放
  \cchessset{resize/type=real,
    resize/width=.95\linewidth}
  \cchessboard[shadow=red!80]
\end{SideBySideExample}

\bigskip

\subsubsection{笔划}

\begin{option}{ opt = scharstroke, desc = {= \meta{棋子字符笔划类型}}, init=bold }
  设置棋子字符笔划类型。
\end{option}

  \begin{Description}[a]
    \item |none|---默认笔划。
    \item |solid|---实线笔划轮廓,笔划内部透明,颜色使用字符颜色。
    \item |white|---实线白色笔划轮廓,笔划内部用白色填充。
    \item |bold|---加粗笔划,颜色使用字符颜色。
    \item |invisible|---隐藏笔划。
  \end{Description}

\begin{SideBySideExample}[frame=single,numbers=left,
                xrightmargin=.58\linewidth,gobble=2]
  \centering
  % 为便排版,进行缩放
  \cchessset{resize/type=real,
    resize/width=.95\linewidth}
  \cchessboard[charstroke=white,
    redupper=red,blkupper=black]
\end{SideBySideExample}

\bigskip

\subsection{打谱}

打谱参数位于|man|子选项中,可以使用|man={label=test}|或
|man/label=test|两种形式进行设置。

\subsubsection{label标签}

\begin{option}{ opt = label, desc = {= \meta{棋谱标签}}, init=init-none }
  设置\env{setcchessman*}打谱环境交叉引用标签。
\end{option}

  该选项目前仅支持\env{setcchessman*}打谱环境。

\subsubsection{每行棋谱步数}

\begin{option}{ opt = nums, desc = {= \meta{每行棋谱步数}}, init=2 }
  设置\tn{printman}输出棋谱时,每行输出棋谱的步数。
\end{option}

  该选项只在\tn{printman}命令中起作用。

\begin{SideBySideExample}[frame=single,numbers=left,
                xrightmargin=.45\linewidth,gobble=2]
  \centering
    \printman{test}
    % 为便排版,进行缩放
    \cchessset{resize/type=real,
             resize/width=0.90\linewidth}
    \begin{setcchessman*}[man/label=test]
      % 残局初始化
      \init{ {{4,0}{K}}, {{2,2}{p}},
             {{1,7}{C}}, {{4,9}{k}} }
      % 添加棋子
      \set{e,d}{P} \set{8,g}{h}
      % 删除棋子
      \del{4,3}
      % 移动棋子
      \mov{C}{b,h}{b,b} \mov{p}{2,2}{3,2}
      \mov{K}{4,0}{4,1} \mov{h}{8,6}{6,7}
      \mov{K}{4,1}{3,1} \mov{h}{6,7}{5,5}
    \end{setcchessman*}
    \begin{flushleft}
      \small
      \printman[man/nums=4]{test}
    \end{flushleft}
\end{SideBySideExample}

\subsection{棋谱缩放}

缩放方式位于|resize|子选项中,可以使用|resize={type=real}|或
|resize/xscale=0.5|两种形式进行设置。

\subsubsection{缩放方式}

\begin{option}{ opt = type, desc = {= \meta{棋谱缩放方式}}, init=none }
  设置最终棋谱输出时的整体缩放方式,目前支持:
\end{option}

  \begin{Description}[a]
    \item |none|---无缩放。
    \item |real|---按指定的width、height、xscale或yscale进行缩放。
  \end{Description}

\bigskip

\subsubsection{横向缩放因子}

\begin{option}{ opt = xscale, desc = {= \meta{x方向缩放系数}}, init=1 }
  设置x方向缩放系数,需要配合|resize|缩放方式参数一起使用。
\end{option}

\begin{SideBySideExample}[frame=single,numbers=left,
                xrightmargin=.48\linewidth,gobble=2]
  \centering
  \cchessboard[resize/type=real,
              resize/xscale=0.5]
\end{SideBySideExample}


\bigskip

\subsubsection{纵向缩放因子}

\begin{option}{ opt = yscale, desc = {= \meta{y方向缩放系数}}, init=1 }
  设置y方向缩放系数,需要配合|resize|缩放方式参数一起使用。
\end{option}

\begin{SideBySideExample}[frame=single,numbers=left,
                xrightmargin=.70\linewidth,gobble=2]
  \centering
  \cchessboard[%
    resize/type=real,
    resize/yscale=0.5]
\end{SideBySideExample}

\bigskip

\subsubsection{横向尺寸}

\begin{option}{ opt = width, desc = {= \meta{宽度}}, init=init-none }
  设置输出宽度,需要配合|resize|缩放方式参数一起使用。
\end{option}

\begin{SideBySideExample}[frame=single,numbers=left,
                xrightmargin=.65\linewidth,gobble=2]
  \centering
  \cchessboard[%
    resize/type=real,
    resize/width=7cm]
\end{SideBySideExample}

\bigskip

\subsubsection{纵向尺寸}

\begin{option}{ opt = height, desc = {= \meta{高度}}, init=init-none }
  设置输出高度,需要配合|resize|缩放方式参数一起使用。
\end{option}

\begin{SideBySideExample}[frame=single,numbers=left,
                xrightmargin=.65\linewidth,gobble=2]
  \centering
  \cchessboard[%
    resize/type=real,
    resize/height=6cm]
\end{SideBySideExample}

  \textbf{\textsf{说明: }}缩放纵/横参数可以混合使用。

\bigskip

% \bigskip

% \title{
%   \pkg{chinesechess} package for writing Chinese chess
%   \rlap{\makebox[4cm][r]{
%     \normalsize $\Longrightarrow$ \color{red}
%     \protect\hyperlink{zh}{中文版本}
%     \protect\hypertarget{en}{}
%   }}
% }
% \author{Nan Geng \texttt{<nangeng@nwafu.edu.cn>}}
% \date{\today\qquad \vers}
% \maketitle
%
% \section{Introduction}
%
% \section{Interface}
%
% \section{options}

\end{documentation}

\end{document}
